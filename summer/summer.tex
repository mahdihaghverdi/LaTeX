\documentclass[12pt]{article}

\usepackage{hyperref}
\hypersetup{
	colorlinks=true,
	linkcolor=blue,
	filecolor=magenta,      
	urlcolor=cyan,
	pdftitle={Overleaf Example},
	pdfpagemode=FullScreen,
}

\urlstyle{same}

\title{Summer plans \& works to do}
\author{Mahdi Haghverdi}
\date{}

\begin{document}
	\maketitle
	\tableofcontents
	\newpage
	
	\begin{abstract}
		Well summer, the season of progress. The season to learn new things and focus on what you really want and build your career.

		After spending two semester in the university, it is clear that \underline{meanwhile} studying, learning \textbf{new}  things, is hard! It's possible but it's hard. So summer is the time which I have to value a lot and do my best to use it right.
		
		At the time of writing this, there are 5 days remaining of the 1401 year, a long and hard year both for Iran, my love, and myself. I learned and grew a lot during this year. I spend a whole semester in university, experienced a really \textit{Randomly-Generated-but-Related} journey. Studied a lot, learned how to study well and meet many new and nice people.
		
		But after all, I have to create a path to my actual career! After working and searching I finally made my decision and chose: Data Science. I really like this field, it's very amazing but hard to learn :) Besides this, I wanna learn backend engineering as well, 'cause it is much faster (faster to get result, lol) and has a slightly smaller learning curve.
		
		The deep learning course needs a book nearby so the first book is \textit{Deep Learning for Coders With Fastai and PyTorch}.
		Refreshing my Python skills is something that I never miss and enjoy a lot, and the most recent book which is published is \textit{Python Distilled} by \textit{David Beazly}. But I have to make a dicision
		Watching video courses and reading books are the main activities of the summer, but without spending or more precisely, acquiring time to rest, nobody can learn anything; listening to music and podcast and watching movies are the main non-active things and biking is the main physical activity of summer.
				
	\end{abstract}

	\section{Courses}
		These are the video courses of summer
		
		\subsection{Deep learning}
			The nice deep learning course is at \url{https://course.fast.ai/}.
			
			\textbf{\textit{Practical Deep Learning}} \\
			\textit{A free course designed for people with some coding experience, who want to learn how to apply deep learning and machine learning to practical problems.}
			
			\textit{This free course is designed for people (and bunnies!) with some coding experience who want to learn how to apply deep learning and machine learning to practical problems.}
			
			Each lesson of the course has a video and a dedicated page of the website, like \href{https://course.fast.ai/Lessons/lesson1.html}{Lesson 1}. 
			
			Notes:
			\begin{itemize}
				\item Each video pretty long, on average they last one hour and 30 minutes! The longer the video, the more knowledge they contain AND the more attention they need AND the more practice as well.
				
				\item Each lesson has a \textit{How to complete lesson N} section (like \url{https://course.fast.ai/Lessons/lesson1.html#how-to-complete-lesson-1},) in which says: \textit{As well as watching the video and working through the notebooks, you should also read the relevent chapter(s) of the fast.ai book, Practical Deep Learning for Coders. Each lesson will tell you what chapter you need to read, just below the video}
			\end{itemize}
		
		    After all, the dedicated time and effort must be huge.
		    	
		\subsection{FastAPI}
			\textit{FastAPI framework, high performance, easy to learn, fast to code, ready for production}
			
			We all know FastAPI and there is no need for more introduction.
			The course I wanna watch is \href{https://www.google.com/url?sa=t&rct=j&q=&esrc=s&source=web&cd=&cad=rja&uact=8&ved=2ahUKEwjUj9jUhOH9AhUPHOwKHdrGCPcQwqsBegQIChAE&url=https%3A%2F%2Fwww.youtube.com%2Fwatch%3Fv%3D0sOvCWFmrtA&usg=AOvVaw0SzlpIrXRghxcsNXprzEkT}{this course}.
			
			The course is a 19-hour long video which is downloaded and cut already. 
			
			The important things are:
			\begin{itemize}
				\item The course is created one year ago and it is slightly outdated, so the \href{fastapi.tiangolo.com}{documentation} must be read along the course.
				
				\item Course will just show you how to use FastAPI and the real learning happens when I do projects.
				
				\item It's good to define nice projects whenever I came to an idea
				e.g. \textit{The API to send pictures of \url{unsplash.com} to my friends :)}
			\end{itemize}
	
	\section{Books}
		\subsection{Deep Learning for Coders With Fastai and PyTorch}
		\subsection{Python Distilled}
		\subsection{Fluent Python}
		\subsection{Build a Career in Data Science}
		
	\section{Habits}
		\subsection{Podcast}
		\subsection{Music}
		\subsection{Films}
		\subsection{Biking}
%	\section{Skills}
	
\end{document}
