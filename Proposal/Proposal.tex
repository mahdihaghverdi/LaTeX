\documentclass{article}

\usepackage{hyperref}
\hypersetup{
	colorlinks=true,
	linkcolor=blue
}
\usepackage[inline]{enumitem}
\usepackage{xepersian}
\settextfont{XB Yas}

\title{خلاصه‌ی تحقیق}
\author{مهدی حق‌وردی}

\begin{document}

\newcommand{\important}{{\tiny (\textbf{مهم}) }}
\maketitle

\tableofcontents
\newpage

	\begin{abstract}
	این فایل،‌ یک خلاصه‌ی کامل از یک تحقیق آماری‌ست. هدف از این تحقیق بررسی عمیقی بر دیتا‌های ذخیره‌ شده‌ از دانشجویان، سلف دانشگاه و رستوران‌های یاس در پایگاه‌‌های اطلاعاتی دانشگاه اصفهان است.
	
	هدف اصلی این تحقیق،‌ بدست آوردن \textbf{دانش} از اطلاعات ذخیره‌ شده است. این دانش به چند قسمت، تقسیم می‌شود:
	\begin{enumerate}
		\item 
		بررسی کامل آماری اطلاعات ذخیره شده.
		\item
		 استفاده از برنامه‌نویسی برای بررسی داده‌ها و \lr{data analysis}.
		\item \important
		 پیش‌بینی مواردی همچون:
		\begin{enumerate}
			\item
			پیش‌بینی تعداد‌ غذا‌هایی که باید در رستوان طبخ شود (که با خود پیش‌بینی‌های زیر را به همراه دارد:)
			\begin{itemize}
			    \item 
			    پیش‌بینی مقدار مصرف مواد اولیه‌ رستوان‌ها.
			    \item 
			    پیش‌بینی میزان نیاز به مواد اولیه‌ی رستوران‌‌ها.
			\end{itemize}
		    \item 
		    پیش‌بینی هزینه‌های هر رستوران.
		\end{enumerate}
	    \item 
	    یافتن الگو‌های رفتاری دانشجویان در مراجعه به رستوان‌‌های یاس.
		\item \important
			برآورد و پیش‌بینی میزان یارانه‌ی اختصاص یافته‌ دولت به دانشجویان استفاده کننده از رستوان‌های یاس.
			
			این بخش یکی از اصلی‌ترین اهداف این مقاله به شمار می‌رود و جرقه‌ی اولیه این تحقیق هم با همین هدف خورده‌ شد:
		\begin{enumerate}
			\item 
			بررسی اولیه‌ی خرج‌ شدن یارانه‌های دولتی برای هر دانشجو در رستوان‌های یاس.
			\item 
			یافتن پر خرج‌ترین دانشجو‌ها (برای دولت) و یافتن بهترین راه‌حل برای تخصیص  عادلانه‌ی یارانه‌ی دولت به دانشجویان.
			\item 
			پیشنهاد و بررسی راه‌حل افزایش یا کاهش یارانه‌ی اختصاص یافته به دانشجو‌ها بر اساس الگو‌های رفتاری آنها در استفاده از رستوان‌های یاس.
		\end{enumerate}
	    \item 
	    برآورد هزینه‌ای که دانشگاه بابت رستوان‌های یاس می‌پردازد.
	\end{enumerate}
	\end{abstract}
	\newpage

	\section{بررسی کامل آماری اطلاعات ذخیره شده}
		این تحقیق چون ذاتن یک کار آماری‌ست (و جرقه‌ی شروع آن بخاطر کلاس \textit{آمار و احتمال مهندسی} خورده شده) نیاز به این داشت که یک \textit{مطالعه‌ی آماری} صورت بگیرد. همچنین، چون ماهیت و ذات مواردی که در چکیده گفته شد و توضیح آنها در بخش‌های دیگر به طور مفصل آورده‌ خواهد شد، برای تکمیل و دست یافتن به پاسخ‌های صحیح، نیاز به علم آمار و احتمال و استفاده از آن برای به سرانجام رساندن در این تحقیق احساس می‌شود.
		
		من آنچنان دانشی از علم آمار و احتمال و خصوصا کاربرد‌های آن در انجام چنین پروژه‌های تحقیقاتی ندارم و همچنین تاکنون، چُنین تحقیقی را انجام نداده‌ام و اصلا با چنین فضایی آشنا نیستم و بابت اینهاست که از شما خواهش کردم که من را در انجام این تحقیق یاری بفرمایید. 
		
		به طور کلی، نمیتوان این بخش \textit{بررسی کامل آماری اطلاعات ذخیره شده} را یک بخش جدا در نظر گرفت؛ تمام این تحقیق از علم آمار و بررسی داده‌ها استفاده می‌کند و یکی از اهداف اصلی‌‌‌ آن استخراج \textbf{دانش} از اطلاعات تل‌انبار شده‌ی پایگاه‌‌های داده است.
		

	\section{استفاده‌ از برنامه‌نویسی برای بررسی داده‌ها و \lr{data analysis}}
		رشته‌ی من مهندسی کامپیوتر است و یکی از علوم و شغل‌‌هایی که امروزه بسیار جذاب است و در رشته‌ها داغ شده، و نیاز به دانش عمیق و همچنین انجام چنین پروژه‌‌هایی دارد، \textit{علم داده‌} است.
		
		همانطور که خودتان هم با این علم آشنایی دارید، یکی از قسمت‌‌های مهم و بزرگ‌ آن برنامه‌نویسی است، و بیشترین میزان یادگیری (خصوصا در برنامه‌نویسی) \textbf{تجربه} و \textbf{استفاده} از آن است. من هم این فرصت را در این پروژه دیدم که درگیر چالش‌های واقعی علم و مهندسی داده هم در قسمت علمی و تئوری آن، و هم در قسمت عملی و برنامه‌نویسی آن بشوم، و اندک تجربه‌ای در کار کردن با دیتا‌های با حجم زیاد داشته باشم.
		
	\section{پیش‌بینی‌ها}
	    از مهم‌ترین قسمت‌ها و نقاط عطف این تحقیق بدست اوردن پیش‌بینی‌هایی‌ست که در قسمت چکیده ذکر شدند و به سه قسمت زیر تقسیم می‌شوند:
	    \subsection{پیش‌بینی‌های مربوط به رستوران}
	    	این تحقیق حول رستوران‌‌های یاس می‌چرخد و طبیعی‌ست که باید راجع به دیتایی که از خود رستوران‌‌ها ذخیره شده تحقیق و بررسی صورت بگیرد و بتوان کاری برای بهتر شدن خدمات و همچنین متعادل شدن و کمتر شدن هزینه‌های آنها انجام داد.
	    	
	    	به عنوان مثال، رستوران‌های \lr{Mc Donalds} که از بزرگ‌ترین رستوران‌های زنجیره‌ای دنیا هستند، از هوش مصنوعی برای پیش‌بینی و یافتن الگوهایی از دیتا‌های رستوران‌های خود استفاده میکنند، برای مثال:
	    	\begin{enumerate}
	    		\item 
	    		میزان و تعداد غذای‌های مختلفی که در ماه‌‌ها و فصول مختلف سال مصرف می‌شوند. 
	    		
	    		
	    		با تحلیل این اطلاعات می‌توان ذائقه‌ی مشتری‌ها را پیدا کرد و غذاهایی که در ماه‌‌های مختلف کمتر مصرف می‌شوند را از منو حذف نمود و منوی بهتری را به مشتری‌‌ها ارائه کرد و همچنین مواد اولیه آنها را تبعا نخرید و از تلف شدن پول جلوگیری نمود. این بحث در رابطه‌ها غذاهای پر طرفدار هم کاملا صدق می‌کند.
	    		
	    		\item 
	    		میزان و پیش‌‌بینی دقیق میزان مواد اولیه لازم برای این ماه یا فصل
	    		
	    		
	    		آنها میتوانند با پیش‌بینی‌های دقیق میزان نیاز به فلان مواد اولیه (مثلا میزان پنیر برای همبرگر‌های مک دونالدز در ماه دسمابر) را بدست آورده و با کمترین میزان هدر رفت مواجه شوند، و همچنین خود را برای کمبود‌هایی که گاهی ممکن است در بازار مواد اولیه بوجود بیاید، آماده کنند.
	    	\end{enumerate}
	    \subsection{پیش‌بینی‌های مربوط به دانشجویان}
	    	یکی از قسمت‌های دیگری که برای رستوران‌ها هم بسیار کمک کننده است، پیش‌بینی‌های مربوط به الگو‌های رفتاری و میزان خرید دانشجویان از رستوران‌هاست.
	    	
	    	یکی از قسمت‌های اصلی این قسمت، بدست آوردن پیش‌بینی‌های دقیق از میزان خرید غذا‌های مختلف توسط دانشجویان است و سپس استفاده از آنها برای تعیین تعداد \textit{فلان} غذایی که در رستوران‌های یاس باید طبخ شود.
	    	
	    	این کار چند فایده‌ دارد:
	    	\begin{enumerate}
	    		\item 
	    		غذای اضافی پخته نمی‌شود و اسرافی صورت نمیگیرد.
	    		\item 
	    		غذایی کم نمی‌آید.
	    		
	    		بارها برای بسیاری از دانشجویان پیش آمده که غذایی که سفارش داده‌اند تمام شده و مجبور به تهیه غذایی دیگری شده‌اند.
	    		\item 
	    		می‌توان از این اطلاعات برای پیش‌بینی میزان مواد اولیه‌ای که باید خریداری شود، استفاده نمود.
	    	\end{enumerate}
	    \subsection{پیش‌بینی‌های مالی}
	    	پیش‌بینی‌های مالی هم، همانطور که از اسمش پیداست مربوط میزان دخل و خرج رستوران‌های یاس می‌شود و این قسمت تلاش می‌کند که میزان دخل و خرج را متعادل و بهبود ببخشد.
	    	
	\section{یافتن الگو‌های رفتاری دانشجویان در مراجعه به رستوران‌های یاس}
		این قسمت اندکی بالاتر بحث شد ولی در این قسمت جنبه‌های دیگری از این مبحث هم بررسی می‌شوند.
		
		برای مثال تفاوت تعداد مراجعه‌ به رستوران یاس ۱ و یا یاس ۲. فهمیدن این باعث می‌شود که تمام ابزار‌آلاتی که در این رستوران‌‌ها استفاده می‌شوند (مثل
		\begin{enumerate*}
			\item ظرف و طروف مورد استفاده
			\item لوازم آشپزی مورد استفاده 
			\item دیگر موارد جانبی مثل سفره‌های مصرفی و...
		\end{enumerate*})
	به اندازه کافی باشند تا نه زیاد باشند و نه کم بیایند (بوده تجربیاتی که صف انتظار غذا، منتظر شسته و تمیز شدن سینی و بشقاب و قاشق چنگال شده‌اند.)
	
	مورد دیگر پیش‌بینی میزان خرید دانشجو‌هاست، اما از چند نظر:
	\begin{enumerate}
		\item 
		تعداد غذا.
		
		بالاتر بحث شد اما تعداد غذا از مهم‌ترین و مستقیم‌ترین مواردی‌ست که به رفتار دانشجویان و خرید آنها بستگی دارد. این کمک می‌کند که تعداد غذای اندازه‌ای طبخ شود که از اسراف و همچنین از کمبود آن غذا جلوگیری شود.
		
		\item تعداد کل غذاهایی که در روز‌های خاصی خریداری می‌شوند
		
		این مسئله‌ی مهمی‌ست. اکثر شنبه‌ها رستوران‌های یاس ۱ و ۲ بسیار بسیار شلوغ می‌شوند، اما خب چرا؟
		برای خرید غذاهای سلف دانشگاه اصفهان از وبسایت کالینان، زمان مشخصی وجود دارد، مثلا اگر از روز پنچ‌شنبه گذشت و شما برای شنبه‌ی خود غذا نخریدید، دیگر نمیتوانید از سلف غذا بخرید و مجبور می‌شوید که از یاس خرید کنید، و چون اکثر قریب به اتفاق دانشجو‌‌ها این قضیه از یادشان می‌رود، شنبه‌ها راهی یاس می‌شوند و یاس خیلی شلوغ می‌شود. می‌توان چنین روز‌هایی را شناسایی کرد و برنامه ریزی کرد که یاس زودتر باز بشود و همچنین به دانشجویان اطلاعات رسانی شود که از شلوغی بیش از حد یاس (گاهی منتظر ماندن برای غذا از یک ساعت و نیم هم تجاوز می‌کند!!) جلوگیری شود.
	\end{enumerate}
	
	\section{برآورد و پیش‌بینی میزان یارانه‌ی اختصاص یافته‌ به دانشجویان}
		ایده‌ی اصلی و مهم‌ترین قسمت‌ این تحقیق همین قسمت است. اگر با اطلاعاتی که در اختیار داریم، پیش‌بینی کنیم که برای یک ترم، چه اندازه یارانه‌ صرف دانشجو‌هایی که از سلف و رستوران‌های یاس استفاده می‌کنند، می‌شود، می‌توان با اصلاح الگوی یارانه‌ای دولت (شبیه به قبوض آب و... که به صورت پلکانی برای مشترکین پر مصرف، یارانه‌ی اختصاص یافته کمتر و برای مشترکین کم مصرف، قبض رایگان می‌شود،) میزان‌ هزینه‌‌های دولت را کاهش و یارانه را به صورت عادلانه‌تری بین دانشجویان تقسیم نمود.
		
		
		اما ممکن است برای شما سوالی پیش  آمده باشد که \textit{چرا یارانه‌ای که هم اکنون دولت به دانشجویان می‌دهد، به صورت عادلانه نیست؟}
		پاسخ ساده‌ است، بیایید طریقه‌ی تخصیص یارانه را قبل از پلکانی کردن قبوض بررسی کنیم: ما دو خانواده داریم، یکی با دو بخاری میتواند خانه‌ی خود را گرم کند، اما دیگری چون خانه‌ی بزرگ‌تری دارد باید از ۵ بخاری استفاده کند تا خانه‌ی خود را گرم کند؛ خب بسیار واضح و مشخص است که هر چقدر بیشتر گاز یارانه‌ای مصرف کند، یارانه‌ی بیشتری هم صرف او می‌شود. اما همین قضیه بعد از پلکانی شدن قبوض چگونه است؟ ابتدا یک سطح پایه‌ برای تمام مشترکین تعریف می‌شود و سپس هر کس بیشتر از این سطح پایه مصرف کند، مبلغ قبض او به صورت پلکانی افزایش می‌یابد یا به صورت دیگر \underline{یارانه‌ی کمتری به او تعلق می‌گیرد.} 
		
		دقیقا چنین حالتی برای دانشجویان دانشگاه هم صدق می‌کند، هر دانشجویی بیشتر به یاس مراجعه کند، تبعاً یارانه‌ی بیشتری هم به او تعلق می‌گیرد! و این باعث می‌شود که یارانه‌ای که دولت به این قسمت اختصاص می‌دهد، به صورت عادلانه بین دانشجویان تقسیم نشود. 
		
		برای تقسیم عادلانه‌ی یارانه‌ می‌توان راه‌های مختلفی را (بعد از بررسی دیتا و بدست‌ آوردن پیش‌بینی‌های هر روش) پیشنهاد داد و عملی کرد، اما هم‌اکنون من صرفا می‌توانم همین حالتی که دولت برای قبوض استفاده می‌کند را پیشنهاد کنم:‌ هر کسی بیشتر از رستوران یاس خرید کند (بعد از رد شدن حد پایه‌ی یارانه‌ی دولت) به صورت پلکانی یارانه‌ی اختصاص یافته به او کاهش یابد. (همچنین میتوان یارانه‌ی اختصاص یافته‌ به کسانی که کمتر از یاس استفاده میکنند یا بیشتر هم کرد، شبیه به رایگان کردن قبوض مشترکینی که مصرف پایینی دارند.)
		
		شاید این بحث پیش‌ بیاید که دارندگی و برازندگی! و با این توجیح که \textit{هر چقدر هم یارانه بگیرم، موقع آزادسازی مدرک همان را پرداخت میکنم!} عده‌ای با این کار مخالفت کنند، اما این توجیح چندین ایراد دارد:
		\begin{enumerate}
			\item 
			این حرف دقیقا مثل حرف کسانی‌ست که ماشین، جلوی در خانه‌‌شان و کوچه را با شلنگ آب شهری می‌شویند! شنیدید که می‌گویند \textit{پولش رو میدم!} اما هر طور که به قضیه نگاه کنیم، بالاخره اسراف صورت می‌گیرد!
			\item 
			چند درصد افراد مدرک خود را در سال آزاد می‌کنند؟ چند درصد از این پول‌ها به خزانه یا ... باز میگردد؟
			\item 
			وضعیت چندین ساله‌ی کشور را ببینید، آیا پول ارزشی دارد؟ آیا چنین نیست کسی که طلبکار است، هر چقدر از زمان وصول پولش بگذرد، ارزش پول او کمتر و کمتر می‌شود؟! و کسی که سرمایه‌دار است (سرمایه‌هایی مثل ملک و ماشین و ...) هر ساله پول‌دارتر می‌شود؟!
			این بدهکاری ما دانشجویان به دولت هم دقیقا چنین است، میزان ارزشی که چهار سال قبل خرج ما شده، بسیار بیشتر از اکنون است!
		\end{enumerate}
	پس هر چقدر این یارانه عادلانه‌تر بین دانشجویان تقسیم بشود، یا از اسراف آن جلوگیری شود، بیت المال بیشتر حفظ می‌شود و می‌توان از آن استفاده درست‌تر و ماندگارتری از احساس سیری کرد.
	
	برای مثال، اگر دانشگاه تمام یارانه‌ی صرفه‌جویی شده در یک یا چند ترم را ذخیره کند، آیا نمی‌تواند به خوابگاه‌های دانشجویی رسیدگی کند و آنها را بازسازی کند؟!
	
	\section{برآورد هزینه‌ای که دانشگاه بابت رستوان‌های یاس می‌پردازد}
		هر مجموعه‌ای که در دانشگاه (یا در جامعه) فعالیت می‌کند، تبعا دارای هزینه‌هایی‌ست که باید آنها را پرداخت کرد. رستوان‌های یاس هم از این قاعده مستثنا نیستند و قطعا خرج، و با توجه به حجم مراجعه و اندازه‌ آنها باید خرج زیادی داشته باشند.
		
		اگر دخل و خرج چنین مجموعه‌هایی بررسی، پیش‌بینی و مدیریت درستی نداشته باشد، هم می‌تواند فساد برانگیز باشد و هم سر به فلک بکشد. بنابراین باید از داده‌های آماری دخل و خرج رستوان استفاده کرد بررسی خوبی روی آنها انجام داد.
		
	\section*{صحبت پایانی}
		بعد از امتحان آمار که با شما صحبت کردم و گفتید یک خلاصه به شما بدهم، من ساختار کلی این خلاصه را نوشتم ولی خب امتحان‌های سختی و زیادی داشتیم که دیگر نتوانستم که زودتر کاملش کنم و خواهشمندم عذرخواهی من را بابت این تاخیر بپذیرید.
\end{document}
