\chapter{Books}
		\begin{sansserif}
			\para{You may argue me about what I want to say, however I found it true: \say{If you want to learn \textit{something}, the real source is the books about it.} This is absolutely true. No video course and no article and nothing else may contain the most valuable information about something. All the video courses and simple article ARE come from the books.}
			
			\para{In this chapter I may introduce many invaluable books (most technical books but it may contain other books as well.)}
			
			\noindent Each section of this chapter has the name of the book which is a link to the post of that book in the store: \url{Skybooks.ir}, where I buy the books from. Then there maybe some tags under the name of the books, the whole tags are categorized in three categories:
			\begin{enumerate}
				\item Reading priority
				\begin{itemize}
					\item \mustread
					\item \betteread
					\item \laterread
				\end{itemize}
				\item Buying considerations:
				\begin{enumerate}
					\item Number of pages
					\begin{itemize}
						\item \nearonethpages
						\item \nearsevenhpages
						\item \nearfivehpages
						\item \neartwohpages
					\end{itemize}
					\item Price:
					\begin{itemize}
						\item \pricenearfive{number}
						\item \pricenearfour{number}
						\item \pricenearthree{number}
						\item \priceneartwo{number}
						\item \pricenearone{number}
					\end{itemize}
				\end{enumerate}
				\item Book read: \bookread
				\item Printed or not: \printed
			\end{enumerate}
		
			\para{The last section of this chapter will discuss how and when the books will be read by me :D}
		\end{sansserif}
			
		\clearpage
		\section{Fluent Python}
		\label{sec:fluent-python}
		\begin{inparadesc}
			\item \betteread
			\item \nearonethpages
			\item \pricenearfive{505}
		\end{inparadesc}
		\vspace{3mm}
		
		\bookcover{./images/fluent}
		
		\para{Book link: \url{https://skybooks.ir/products/Fluent-Python}}
		
		\para{Python’s simplicity lets you become productive quickly, but often this means you aren’t using everything it has to offer. With the updated edition of this hands-on guide, you’ll learn how to write effective, modern Python 3 code by leveraging its best ideas. Don’t waste time bending Python to fit patterns you learned in other languages. Discover and apply idiomatic Python 3 features beyond your past experience. Author Luciano Ramalho guides you through Python’s core language features and libraries and teaches you how to make your code shorter, faster, and more readable.} 
		
		\noindent Featuring major updates throughout the book, Fluent Python, second edition, covers:
		\begin{enumerate}
			\item Special methods: The key to the consistent behavior of Python objects
			\item Data structures: Sequences, dicts, sets, Unicode, and data classes
			\item Functions as objects: First-class functions, related design patterns, and type hints in function declarations
			\item Object-oriented idioms: Composition, inheritance, mixins, interfaces, operator overloading, static typing and protocols
			\item Control flow: Context managers, generators, coroutines, async/await, and thread/process pools
			\item Metaprogramming: Properties, attribute descriptors, class decorators, and new class metaprogramming hooks that are simpler than metaclasses
		\end{enumerate}
	
		\clearpage
		\section{Inside the Python Virtual Machine}\label{sec:py-venv}
		\begin{inparadesc}
			\item \mustread
			\item \neartwohpages
			\item \pricenearone{88}
			\item \printed
		\end{inparadesc}
		\vspace{3mm}
		
		\bookcover{./images/insidepyvenv}
		
		\para{Book link: \url{https://skybooks.ir/products/Inside-The-Python-Virtual-Machine}}
		
		\para{Inside the Python Virtual Machine provides a guided tour under the covers of the Python interpreter for the curious pythonista. It attempts to show the user what happens from the moment the user executes a piece of Python code to the point when the interpreter returns the result of executing the piece of code.}
		
		\para{This book will provide the readers with an understanding of the various processes that go into compiling and executing a python program removing most of the mystery surrounding how the python interpreter executes source code.} 
		
		\para{The books starts out with a description of the compilation phase with emphasis on the less generic parts of the compilation phase. It then proceeds to discuss python objects and their implementation in CPython.  This is followed by a discussion of various objects types that are central to the interpreter such as frame objects and code objects. The process of evaluating code objects by the interpreter loop is also discussed as well as how to extend the Python programming language with your own constructs.}
		
		\clearpage
		\section{CPython Internals}
		\label{sec:py-internals}
		
		\begin{inparadesc}
			\item \mustread
			\item \nearfivehpages
			\item \pricenearthree{209}
			\item \printed
		\end{inparadesc}
		\vspace{3mm}
		
		\bookcover{./images/internals}
		
		\para{Book link: \url{https://skybooks.ir/products/CPython-Internals}}
		
		\para{CPython Internals: Your Guide to the Python 3 Interpreter.}
		
		\para{Are there certain parts of Python that just seem like magic? Once you see how Python works at the interpreter level, you’ll be able to optimize your applications and fully leverage the power of Python.}
		
		\para{In CPython Internals, you’ll unlock the inner workings of the Python language, learn how to compile the Python interpreter from source code, and cover what you’ll need to know to confidently start contributing to CPython yourself!}
		\clearpage
		\section{Docker Deep Dive}
		\label{sec:docker-deep-dive}
		
			\begin{inparadesc}
				\item \betteread
				\item \neartwohpages
				\item \priceneartwo{155}
			\end{inparadesc}
			\vspace{3mm}
			
			\bookcover{./images/docker}
			
			\para{Book link: \url{https://skybooks.ir/products/Docker-Deep-Dive}}
			
			\para{Most applications, even the funky cloud-native microservices ones, need high-performance, production-grade infrastructure to run on. Having impeccable knowledge of Docker will help you to thrive in the modern cloud-first world. With this book, you'll gain the skills you need to work with Docker and its containers.}
			
			\para{The book begins with an introduction to containers and explains its functionality and application in the real world. You'll then get an overview of VMware, Kubernetes, and Docker and learn to install Docker on Windows, Mac, and Linux. Once you've understood the Ops and Dev perspective of Docker, you'll be able to see the big picture and understand what Docker exactly does. The book then turns its attention to the more technical aspects, guiding your through practical exercises covering Docker engine, Docker images, and Docker containers. You'll learn techniques for containerizing an app, deploying apps with Docker Compose, and managing cloud-native applications with Swarm. You'll also build Docker networks and Docker overlay networks and handle applications that write persistent data. Finally, you'll deploy apps with Docker stacks and secure your Docker environment.}
			
			\para{By the end of this book, you'll be well-versed in Docker and containers and have developed the skills to create, deploy, and run applications on the cloud.}
			
			\noindent What you will learn:
			\begin{itemize}
				\item Become familiar with the applications of Docker and containers
				\item Discover how to pull images into Docker host's local registry
				\item Find out how to containerize an app
				\item Build and test a Docker overlay network in the swarm mode
				\item Use Docker compose to deploy and manage multi-container applications
				\item Securely share sensitive data with containers and Swarm services
			\end{itemize}

		\clearpage
		\section{Docker in Action}
		\label{sec:docker-in-action}
		
			\begin{inparadesc}
				\item \mustread
				\item \nearfivehpages
				\item \priceneartwo{189}
				\item \printed
			\end{inparadesc}
			\vspace{3mm}
			
			\bookcover{./images/dockeri}
			
			\begin{epigraphcenter}{11cm}
				\begin{sansserif}
					
				\end{sansserif}
				\epigraph{
				\begin{sansserif}
					\say{Jeff and Stephen took their battle-hardened experience and updated this already great book with new details and examples.}
			\end{sansserif}}{From the Foreword by Bret Fisher, Docker Captain and Container Consultant}
			\end{epigraphcenter}
			\para{Book link: \url{https://skybooks.ir/products/Docker-in-Action}}
			
			\para{\textit{Docker in Action, Second Edition} teaches you the skills and knowledge you need to create, deploy, and manage applications hosted in Docker containers. This bestseller has been fully updated with new examples, best practices, and a number of entirely new chapters.}
			
			\para{\textbf{about the technology}}
			\para{The idea behind Docker is simple—package just your application and its dependencies into a lightweight, isolated virtual environment called a container. Applications running inside containers are easy to install, manage, and remove. This simple idea is used in everything from creating safe, portable development environments to streamlining deployment and scaling for microservices. In short, Docker is everywhere.}
			
			\para{\textbf{about the book}}
			\para{\textit{Docker in Action, Second Edition} teaches you to create, deploy, and manage applications hosted in Docker containers running on Linux. Fully updated, with four new chapters and revised best practices and examples, this second edition begins with a clear explanation of the Docker model. Then, you go hands-on with packaging applications, testing, installing, running programs securely, and deploying them across a cluster of hosts. With examples showing how Docker benefits the whole dev lifecycle, you’ll discover techniques for everything from dev-and-test machines to full-scale cloud deployments.}
		
		\clearpage
		\section{Zero to One}
		\label{sec:zero-to-one}
		
			\begin{inparadesc}
				\item \laterread
				\item \neartwohpages
				\item \pricenearone{110}
			\end{inparadesc}
			\vspace{3mm}
			
			\bookcover{./images/zero}
			
			\para{Book link: \url{https://skybooks.ir/products/Zero-to-One}}
			
			\para{Thiel starts from the bold premise that we live in an age of technological stagnation, even if we're too distracted by our new mobile devices to notice. Progress has stalled in every industry except computers, and globalization is hardly the revolution people think it is. It's true that the world can get marginally richer by building new copies of old inventions, making horizontal progress from "1 to n." But true innovators have nothing to copy. The most valuable companies of the future will make vertical progress from "0 to 1," creating entirely new industries and products that have never existed before. Zero to One is about how to build these companies. Tomorrow's champions will not win by competing ruthlessly in today's marketplace. They will escape competition altogether, because their businesses will be unique. In today's post-internet bubble world, conventional wisdom dictates that all the good ideas are taken, and the economy becomes a tournament in which everyone competes to reach the top. Zero to One shows how to quit the zero-sum tournament by finding an untapped market, creating a new product, and quickly scaling up a monopoly business that captures lasting value. Planning an escape from competition is essential for every business and every individual, not just for technology startups. The greatest secret of the modern era is that there are still unique frontiers to explore and new problems to solve. Zero to One shows how to pursue them using the most important, most difficult, and most underrated skill in every job or industry: thinking for yourself"--Provided by publisher.}

		\clearpage
		\section{Microservice APIs: Using Python, Flask, FastAPI, OpenAPI and more}\label{sec:microservice-python}
		
			\begin{inparadesc}
				\item \betteread
				\item \nearfivehpages
				\item \priceneartwo{250}
			\end{inparadesc}
			
			\bookcover{./images/micro}
			
			\begin{epigraphcenter}{11cm}
				\epigraph{
					\begin{sansserif}
						\say{An insightful guide for creating REST and GraphQL APIs, with neat examples using FastAPI and Flask. The service implementation patterns chapter is a must-read for every developer.}
					\end{sansserif}
				}{William Jamir Silva, Adjust}
			\end{epigraphcenter}
		
			\para{Book link: \url{https://skybooks.ir/products/Microservice-APIs}}
			
			\begin{sansserif}
				\para{\textbf{Strategies, best practices, and patterns that will help you design resilient microservices architecture and streamline your API integrations.
				}}
			\end{sansserif}
		
			\noindent In Microservice APIs, you’ll discover:
			\begin{itemize}
				\item Service decomposition strategies for microservices
				\item Documentation-driven development for APIs
				\item Best practices for designing REST and GraphQL APIs
				\item Documenting REST APIs with the OpenAPI specification (formerly Swagger)
				\item Documenting GraphQL APIs using the Schema Definition Language
				\item Building microservices APIs with Flask, FastAPI, Ariadne, and other frameworks
				\item Service implementation patterns for loosely coupled services
				\item Property-based testing to validate your APIs, and using automated API testing frameworks like schemathesis and Dredd
				\item Adding authentication and authorization to your microservice APIs using OAuth and OpenID Connect (OIDC)
				\item Deploying and operating microservices in AWS with Docker and Kubernetes
			\end{itemize}
		
			\para{Microservice APIs teaches you practical techniques for designing robust microservices with APIs that are easy to understand, consume, and maintain. You’ll benefit from author José Haro Peralta’s years of experience experimenting with microservices architecture, dodging pitfalls and learning from mistakes he’s made. Inside you’ll find strategies for delivering successful API integrations, implementing services with clear boundaries, managing cloud deployments, and handling microservices security. Written in a framework-agnostic manner, its universal principles can easily be applied to your favorite stack and toolset.}
			
			\para{\textbf{about the technology}}
			\para{Clean, clear APIs are essential to the success of microservice applications. Well-designed APIs enable reliable integrations between services and help simplify maintenance, scaling, and redesigns. This book teaches you the patterns, protocols, and strategies you need to design, build, and deploy effective REST and GraphQL microservices APIs.}
			
			\para{\textbf{about the book}}
			\para{Microservice APIs gathers proven techniques for creating and building easy-to-consume APIs for microservices applications. Rich with proven advice and Python-based examples, this practical book focuses on implementation over philosophy. You’ll learn how to build robust microservice APIs, test and protect them, and deploy them to the cloud following principles and patterns that work in any language.}
		
		\clearpage
		\section{Reading and Writing Plan}
			\noindent For the summer of 1402 (2023) the book which will be read are:
			\begin{enumerate}
				\item \nameref{sec:py-venv}
				
				This book has many to say, \underline{many} new things of CPython virtual machine. I will say how I will treat this book and \nameref{sec:py-internals} books :D
				
				It will be read in Saturdays and Sundays.
				
				\item \nameref{sec:py-internals}
				
				Well, a long and comprehensive course on CPython itself. I asked to writer, Anthony Shaw, how to read this book and he said: Don't rust! just read chapter by chapter and do the exercise of each chapter when  you finished it.
				
				It will be read on Mondays and Tuesdays.
				
				\item \nameref{sec:docker-in-action}
				
				A long book full of really new information! I have not touched docker yet, so this book should be read carefully.
				
				It will be read on Wednesdays and Thursdays.
			\end{enumerate}
			
			\subsection{Additional information about \nameref{sec:py-venv} and \nameref{sec:py-internals} books}
				\para{In order to learn something, the best way is to teach it to others :D Yes this is approved both by experience and academicly researched by scientist which say \say{The learning effectiveness of teaching a thing is 90 \%}. I really wanna learn these two books and in order to achieve that, I will write articles about them and put my articles in \url{https://virgool.io/@liewpl}\footnote{Persian version} and \url{https://medium.com}\footnote{English version}}