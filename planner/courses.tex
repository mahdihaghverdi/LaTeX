\chapter{Courses}
		\begin{sansserif}
			Well, the major part of the summer is the courses, in this chapter I will introduce the courses I want to watch and some useful information about them.\newline
			
			\noindent Eash course is tagged with these:
			\begin{itemize}
				\item Learning curse
				\begin{itemize}
					\item \hard
					\item \easy
				\end{itemize}
				\item Video duration
				\begin{itemize}
					\item \longvideo
					\item \shortvideo
				\end{itemize}
				\item Video Count: \many
			\end{itemize}
		
			\para{Each course will have a subsection names as \say{Timing and Scheduling}. In this subsection I will talk about the difficulty of the course and how I will watch that course starting from summer and afterwards.}
		\end{sansserif}
		
		\clearpage
		\section{FastAPI}
		\epigraph{
			\begin{sansserif}
				\say{FastAPI framework, high performance, easy to learn, fast to code, ready for production}
			\end{sansserif}	
		}{\url{fastapi.tiangolo.com}}
		
		\begin{inparadesc}
			\item \easy
			\item \shortvideo
			\item \many
		\end{inparadesc}
		\vspace{3mm}
		
		\para{We all know FastAPI and there is no need for more introduction.
			The course I wanna watch is \url{https://youtu.be/0sOvCWFmrtA}}
		
		\noindent The course is a 19-hour long video which is downloaded and cut already. The important things are:
		\begin{itemize}
			\item The course is created one year ago and it is slightly outdated, so the documentation\footnote{\url{https://fastapi.tiangolo.com}} must be read along the course.
			
			\item Course will just show you how to use FastAPI and the real learning happens when doing projects.
			
			\item It's good to define nice projects whenever I came to an idea
			e.g. \textit{The API to send pictures of \url{https://unsplash.com} to my friends} :)
		\end{itemize}
			\subsection{Timing and Scheduling}
				\para{This is a course which is pretty long: about 19 hours of content, but it has a noticeable difference with the DL course: it has nearly 150 lessons and the durations of the lessons is short. Also FastAPI is easy, it is so so so much easier that DL. Besides all of this I wanna learn FastAPI fast and through the course away and do small to medium size projects with it. With all this considerations the timing and scheduling of this course will be as follows: I will watch and learn by concept. For example in \underline{week} or \underline{two or three days} I would learn the CRUD\footnote{Create, Read, Update and Delete; It is said to point the basic operations of a web API and database} operations of FastAPI.}
				
				\para{Doing projects with FastAPI is one of the critical and important things which I have to dedicate my time to do. Ideas for projects are much and will come to mind easily but doing them takes time and effort. After all \textbf{I should do projects.}}
				
				\para{I am learning FastAPI because back-end has very nice opportunity to get a job as a junior developer, also in MLOps\footnote{A career which is a mixture of Machine Learning + Development + Operations. It's all about data science but in a more DevOps manner.} I have to know back-end and Ops stuff well ({\small that's why I wanna learn docker as well.})} 