\chapter{Courses}
		\begin{sansserif}
			\para{Well, the major part of the summer is the courses, in this chapter I will introduce the courses I want to watch and some useful information about them.}
		\end{sansserif}
		
		\clearpage		
		\section{Deep learning}
			\epigraph{
				\begin{sansserif}
					The career I chose, \say{The sexiest job of the 21st century,} \say{The best job in America,} \say{One of the newest and coolest jobs in IT world}
				\end{sansserif}
		    }
			
			\para{Well as the quote says, this is exciting, BUT it really needs knowledge and effort to gain, the starting point is here: this nice deep learning course at \url{https://course.fast.ai/}.}
			
			\para{\textbf{Practical Deep Learning}}
			\para{\textit{A free course designed for people with some coding experience, who want to learn how to apply deep learning and machine learning to practical problems.}}
			
			\para{\textit{This free course is designed for people (and bunnies!) with some coding experience who want to learn how to apply deep learning and machine learning to practical problems.}}
			
			\para{Each lesson of the course has a video and a dedicated page of the website, like \href{https://course.fast.ai/Lessons/lesson1.html}{Lesson 1}. }
			
			\noindent Notes:
			\begin{itemize}
				\item Each video is pretty long, on average they last one hour and 30 minutes! The longer the video, the more knowledge they contain AND the more attention they need AND the more practice as well.
				
				\item Each lesson has a \textit{How to complete lesson N} section (like \href{https://course.fast.ai/Lessons/lesson1.html#how-to-complete-lesson-1}{here},) in which says: \textit{As well as watching the video and working through the notebooks, you should also read the relevent chapter(s) of the fast.ai book, Practical Deep Learning for Coders. Each lesson will tell you what chapter you need to read, just below the video}
			\end{itemize}
		
		    After all, \textsf{the dedicated time and effort must be huge.}
		
		\clearpage
		\section{FastAPI}
			\epigraph{
				\begin{sansserif}
					\say{FastAPI framework, high performance, easy to learn, fast to code, ready for production}
				\end{sansserif}	
			}{\href{https://fastapi.tiangolo.com}{fastapi.tiangolo.com}}
			
			\para{We all know FastAPI and there is no need for more introduction.
			The course I wanna watch is \url{https://youtu.be/0sOvCWFmrtA}}
			
			\noindent The course is a 19-hour long video which is downloaded and cut already. The important things are:
			\begin{itemize}
				\item The course is created one year ago and it is slightly outdated, so the \href{https://fastapi.tiangolo.com}{documentation} must be read along the course.
				
				\item Course will just show you how to use FastAPI and the real learning happens when doing projects.
				
				\item It's good to define nice projects whenever I came to an idea
				e.g. \textit{The API to send pictures of \url{https://unsplash.com} to my friends} :)
			\end{itemize}

