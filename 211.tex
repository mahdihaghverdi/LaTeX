\documentclass{article}

\usepackage{multicol}
\usepackage{multirow}
\usepackage{rotating}
\usepackage{calc}
\usepackage[dvips=false,pdftex=false,vtex=false,b5paper]{geometry}
\geometry{
	paper=a4paper,
	inner=16mm,         % Inner margin
	outer=24mm,         % Outer margin
	bindingoffset=10mm, % Binding offset
	top=20mm,           % Top margin
	bottom=28mm,        % Bottom margin
	%showframe,          % show how the type block is set on the page
}
\renewcommand{\arraystretch}{1.5}
\usepackage{xepersian}
\settextfont{XB Yas}

\begin{document}	
 	\begin{sidewaystable}[h]
		\begin{center}	
 			\caption{جدول زمان‌بندی دروس ترم ۴}	
			\begin{tabular}{|c|c|c|c|c|c|c|c|c|c|c|}
				\cline{2-11}
				\multicolumn{1}{c}{}
				& \multicolumn{10}{|c|}{\textbf{ساعات درسی}} \\ \hline
			   \textbf{روز} & ۸:۰۰ تا ۹:۰۰ & ۹:۰۰ تا ۱۰:۰۰ &
     			\multicolumn{2}{|c|}{۱۰:۰۰ تا ۱۲:۰۰} &
     			۱۲:۰۰ تا ۱۳:۰۰ & ۱۳:۰۰ تا ۱۴:۰۰ &
				\multicolumn{2}{|c|}{۱۴:۰۰ تا ۱۶:۰۰} &
				\multicolumn{2}{|c|}{۱۶:۰۰ تا ۱۸:۰۰} \\
				\hline
				\hline
				
				\textbf{شنبه} &
				\multicolumn{2}{|c|}{} &
				\multicolumn{2}{|c|}{تحلیل و طراحی سیستم‌ها، شعرباف} &
				\multicolumn{2}{|c|}{} &
				\multicolumn{2}{|c|}{آز الکتریکی، خیرمند} &
				\multicolumn{2}{|c|}{} \\
				\hline
				
				\textbf{یک‌شنبه} &
				\multicolumn{2}{|c|}{طراحی الگوریتم، کائدی} &
				\multicolumn{2}{|c|}{معماری، بیکی} &
				 & تحلیل، شعرباف &
				\multicolumn{2}{|c|}{} &
				\multicolumn{2}{|c|}{} \\
				\hline
				\textbf{دوشنبه} &
				نظریه زبان‌ها، کائدی & طراحی الگوریتم، کائدی &
				\multicolumn{2}{|c|}{} &
				& معماری، بیکی &
				\multicolumn{2}{|c|}{} &
				\multicolumn{2}{|c|}{آز فیزیک ۲، جلالی} \\
				\hline
				
				\textbf{سه‌شنبه} &
				\multicolumn{2}{|c|}{} &
                \multicolumn{2}{|c|}{} &
				\multicolumn{2}{|c|}{} &
				\multicolumn{2}{|c|}{} &
				\multicolumn{2}{|c|}{} \\
				\hline
				
				\textbf{چهارشنبه} &
				\multicolumn{2}{|c|}{نظریه زبان‌ها، کائدی} &
				\multicolumn{2}{|c|}{} &
				\multicolumn{2}{|c|}{} &
				\multicolumn{2}{|c|}{آز منطقی، آزادی} &
				\multicolumn{2}{|c|}{} \\
				\hline
            \end{tabular}
		\end{center}
 	\end{sidewaystable}
\end{document}