\documentclass{article}

\usepackage{multicol}
\usepackage{multirow}
\usepackage{rotating}
\usepackage{calc}
\usepackage[dvips=false,pdftex=false,vtex=false,b5paper]{geometry}
\geometry{
	paper=a4paper,
	inner=16mm,         % Inner margin
	outer=24mm,         % Outer margin
	bindingoffset=10mm, % Binding offset
	top=20mm,           % Top margin
	bottom=28mm,        % Bottom margin
	%showframe,          % show how the type block is set on the page
}
\renewcommand{\arraystretch}{2}
\usepackage{xepersian}
\settextfont{XB Yas}

\begin{document}	
	\begin{sidewaystable}[h]
		\begin{center}	
			\caption{جدول زمان‌بندی دروس ترم ۴}	
			\begin{tabular}{|c|c|c||c|c||c|c||c|c||c|c|}
				\cline{2-11}
				\multicolumn{1}{c}{}
				& \multicolumn{10}{|c|}{\textbf{ساعات درسی}} \\ \hline
				
				                \textbf{روز} & 
				۸:۰۰ تا ۹:۰۰ & 
				۹:۰۰ تا ۱۰:۰۰ & 
				۱۰:۰۰ تا ۱۱:۰۰ & 
				۱۱:۰۰ تا ۱۲:۰۰ & 
				۱۲:۰۰ تا ۱۳:۰۰ & 
				۱۳:۰۰ تا ۱۴:۰۰ &
				۱۴:۰۰ تا ۱۵:۰۰ & 
				۱۵:۰۰ تا ۱۶:۰۰ & 
				۱۶:۰۰ تا ۱۷:۰۰ & 
				۱۷:۰۰ تا ۱۸:۰۰ \\
				\hline
				\hline
				
				\textbf{شنبه} &
				& & & & & & & & & \\
				\hline
				
				\textbf{یک‌شنبه} &
				& & & & & & & & & \\
				\hline
				
				\textbf{دوشنبه} &
				& & & & & & & & & \\
				\hline
				
				\textbf{سه‌شنبه} &
				& & & & & & & & & \\
				\hline
				
				\textbf{چهارشنبه} &
				& & & & & & & & & \\
				\hline
			\end{tabular}
		\end{center}
	\end{sidewaystable}
\end{document}