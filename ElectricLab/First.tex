\documentclass{article}

\usepackage{rotating}
\usepackage{mathtools}
\usepackage{xepersian}
\settextfont{Yas}

\title{گزارش کار آزمایش اول}
\author{
	گروه: \\
	اریسا احسانی \\
	سید حسین حسینی \\
	مهدی حق‌وردی
}
\renewcommand{\arraystretch}{1.4}

\begin{document}
	\maketitle
	\tableofcontents
	\newpage
	
	\section{آزمایش ۱}
		\begin{equation}
			V = R \cdot I
		\end{equation}
		
		\subsection{مقاومت ۱}		
		    \begin{equation*}
		    	\begin{rcases}
		    		V = 2.2 \, V  \\	
		    		I = 1.05 \, mA
		    	\end{rcases}
	    		\Rightarrow R = \frac{V}{I} \Rightarrow R = 2.09 \, K \Omega
		    \end{equation*}
		\subsection{مقاومت ۲}
		    \begin{equation*}
				\begin{rcases}
					V = 1.89 \, V  \\	
					I = 1.05 \, mA
				\end{rcases}
				\Rightarrow R = \frac{V}{I} \Rightarrow R = 1.8 \, K \Omega
			\end{equation*}
			
		\subsection{مقاومت ۳}
		   \begin{equation*}
				\begin{rcases}
					V = 0.91 \, V  \\	
					I = 0.13 \, mA
				\end{rcases}
				\Rightarrow R = \frac{V}{I} \Rightarrow R = 7 \, K \Omega
			\end{equation*}
		
		\subsection{مقاومت ۴}
		     \begin{equation*}
		 	 \begin{rcases}
		 	    V = 0.91 \, V  \\	
		 		I = 0.91 \, mA
		 	\end{rcases}
		 	\Rightarrow R = \frac{V}{I} \Rightarrow R = 1 \, K \Omega
		 \end{equation*}
		 
	
	\section{آزمایش ۲}
	قانون اهم را برای هر ۴ مقاومت تحقیق نمایید.
	\begin{center}
		\begin{tabular}{|c|c|c|c|c|c|c|c|}
			\hline
			& رنگ اول & رنگ دوم & رنگ سوم & رنگ چهارم & مقدار نامی & مقدار اندازه‌گیری شده & درصد خطا \\
			\hline
			\hline
			مقاومت شماره‌ یک & قرمز & قهوه‌ای & قرمز & طلایی & 
			 \lr{2.1k} & \lr{2.22k} & \lr{5.7 \%}  \\ \hline
			مقاومت شماره‌ دو & قرمز & خاکستری & قرمز & طلایی &
			 \lr{1.8k} & \lr{1.80k} & \lr{0} \\ \hline
			مقاومت شماره‌ سه & آبی & مشکی & قرمز & طلایی &
			 \lr{6.8k} & \lr{6.88k} & \lr{1.1 \%} \\ \hline
		\end{tabular}
	\end{center}

	\section{آزمایش ۳}
		\begin{center}
			\begin{tabular}{|c|c|c|c|c|}
				\hline 
				پتانسیومتر & عدد نوشته شده & دو سر ثابت & یک سر ثابت یک سری متغیر & اندازه‌یک سر ثابت یک سری متغیر \\
				\hline
				\hline
				\lr{pt1} & \lr{203} & \lr{22.4} & \lr{19.1} & \lr{3.6} \\
				\hline
			\end{tabular}
		\end{center}
	
	\section{آزمایش ۴}
		جدول زیر را برای $R = 330$ و $R = 1800$ کامل کنید.
		
		\begin{center}
			\begin{tabular}{|c|c|c|c|c|c|c|c|c|c|}
				\hline
					$V$ & \lr{0} & \lr{1} & \lr{2} & \lr{3} & \lr{4} & \lr{5} & \lr{6} & \lr{7} & \lr{8} \\
				\hline
				$I (R= 330)$ & \lr{0} & \lr{0.3} & \lr{0.62} & \lr{0.9} & \lr{1.2} & \lr{1.5} & \lr{1.8} & \lr{2.1} & \lr{2.4} \\
				\hline
			\end{tabular}
		\end{center}
	
		\subsection{0}
			\begin{equation*}
				\begin{rcases}
					V = 0 \ V  \\	
					I = 0 \ mA
				\end{rcases}
				\Rightarrow R = \frac{V}{I} \Rightarrow R = 0 \, K \Omega
			\end{equation*}
		
		\subsection{1}
		\begin{equation*}
			\begin{rcases}
				V = 1 \ V  \\	
				I = 3.3 \ mA
			\end{rcases}
			\Rightarrow R = \frac{V}{I} \Rightarrow R = 0.3 \, K \Omega
		\end{equation*}
		
		\subsection{2}
		\begin{equation*}
			\begin{rcases}
				V = 2 \ V  \\	
				I = 3.3 \ mA
			\end{rcases}
			\Rightarrow R = \frac{V}{I} \Rightarrow R = 0.6 \, K \Omega
		\end{equation*}
		
		\subsection{3}
		\begin{equation*}
			\begin{rcases}
				V = 3 \ V  \\	
				I = 3.3 \ mA
			\end{rcases}
			\Rightarrow R = \frac{V}{I} \Rightarrow R = 0.9 \, K \Omega
		\end{equation*}
		
		\subsection{4}
		\begin{equation*}
			\begin{rcases}
				V = 4 \ V  \\	
				I = 3.3 \ mA
			\end{rcases}
			\Rightarrow R = \frac{V}{I} \Rightarrow R = 1.2 \, K \Omega
		\end{equation*}
		
		\subsection{5}
		\begin{equation*}
			\begin{rcases}
				V = 5 \ V  \\	
				I = 3.3 \ mA
			\end{rcases}
			\Rightarrow R = \frac{V}{I} \Rightarrow R = 1.5 \, K \Omega
		\end{equation*}
		
		\subsection{6}
		\begin{equation*}
			\begin{rcases}
				V = 6 \ V  \\	
				I = 3.3 \ mA
			\end{rcases}
			\Rightarrow R = \frac{V}{I} \Rightarrow R = 1.8 \, K \Omega
		\end{equation*}
		
		\subsection{7}
		\begin{equation*}
			\begin{rcases}
				V = 7 \ V  \\	
				I = 3.3 \ mA
			\end{rcases}
			\Rightarrow R = \frac{V}{I} \Rightarrow R = 2.1 \, K \Omega
		\end{equation*}
		
		\subsection{8}
		\begin{equation*}
			\begin{rcases}
				V = 8 \ V  \\	
				I = 3.33 \ mA
			\end{rcases}
			\Rightarrow R = \frac{V}{I} \Rightarrow R = 2.4 \, K \Omega
		\end{equation*}
		
		\section{آزمایش ۵}
			\begin{center}
				\begin{tabular}{|c|c|c|c|}
					\hline
					$V$ & $V1$ & $V2$ & $V3$ \\
					\hline
					\hline
					$4 \ V$ & $-0.7$ & $1.19$ & $2.12$ \\
					\hline
					$10 \ V$ & $-1.75$ & $2.96$ & $5.29$ \\
					\hline
				\end{tabular}
			\end{center}
		
		\section{آزمایش ۶}
			\begin{center}
				\begin{tabular}{|c|c|c|c|}
					\hline
					$V$ & \lr{Pt. Max} & \lr{Pt. Min} & \lr{Pt. Avg} \\
					\hline
					$V1$ & $6.16$ & $0$ & $5.5$ \\
					\hline
					$V2$ & $0.84$ & $6.99$ & $1.5$ \\
					\hline
					
				\end{tabular}
			\end{center}
		
		\section{آزمایش ۷}
			\begin{center}
				\begin{tabular}{|c|c|c|c|c|c|c|c|c|c|}
					\hline
					$V$ & $I1$ & $I2$ & $I3$ & $I$ & $\frac{V}{R_1}$ & $\frac{V}{R_2}$ & $\frac{V}{R_3}$ & $R_{eq}$ & $\frac{V}{R_{eq}}$ \\
					\hline
					\hline
					$5 \ V$ & $15.2$ & $8.93$ & $7.35$ & $31.4$ & $0.015$ & $0.009$ & $0.007$ & $0.006$ & $833$ \\
					\hline
					$10 \ V$ & $30.3$ & $17.9$ & $14.7$ & $62.9$ & $0.03$ & $0.017$ & $0.014$ & $0.006$ & $1666.6$ \\
					\hline
					
				\end{tabular}
			\end{center}
\end{document}