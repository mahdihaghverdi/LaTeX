\documentclass[12pt, fleqn]{book}

\usepackage{amsmath, amssymb, amsthm, amsfonts}
\usepackage{hyperref}
\hypersetup{
	colorlinks=true,
	linktoc=all,
	linkcolor=blue,
}

\title{PDF 13 to 16 Integral Formulas}
\author{Mahdi Haghverdi}

\begin{document}
	\maketitle
	\tableofcontents
	
\chapter{PDF 13}\label{pdf13}
	\section{Line Integral}
	\begin{equation}
		\int\limits_C f \, \mathrm{d}s = \int_{a}^{b} f\big(\vec{r}(t)\big) \, \lvert r'(t) \lvert \, \mathrm{d}t
	\end{equation}		

	\section{Physical Aspect}
		\begin{equation}
			m = \int\limits_C \delta (x, y) \, \mathrm{d}s
		\end{equation}
		
		\begin{equation}
			(\bar{x}, \bar{y}) = 
			\begin{cases}
					\bar{x} = \frac{1}{m} \int\limits_C x \, \delta(x, y) \, \, \mathrm{d}s \\
					\bar{y} = \frac{1}{m} \int\limits_C y \, \delta(x, y) \, \, \mathrm{d}s
			\end{cases}
		\end{equation}
	
	\section{Vector Field}
		\begin{equation}
			\int\limits_C F \cdot \mathrm{d}r = \int_{a}^{b} F\big(\vec{r}(t)\big) \cdot r'(t) \, \mathrm{d}t
		\end{equation}
		
		\begin{equation}
			\int\limits_C F \cdot \mathrm{d}r = \int\limits_C \mathtt{p} \, \mathrm{d}x + \mathtt{q} \, \mathrm{d}y + \mathtt{r} \, \mathrm{d}z
		\end{equation}    









\chapter{PDF 14}\label{pdf14}
    \section{Gradient Vector}
		\begin{equation}
			F = \nabla f \Rightarrow
			\begin{cases}
				p = \frac{\partial f}{\partial x} \\
				q = \frac{\partial f}{\partial y} \\
				r = \frac{\partial f}{\partial z} 
			\end{cases}
		\end{equation} 
		
		\begin{equation}
			\int\limits_C F \cdot \mathrm{d}r = \int_{a}^{b} \nabla f \cdot \mathrm{d}r = f\big(\vec{r}(b)\big) - f\big(\vec{r}(a)\big)
		\end{equation}     
		
		\begin{equation}
			\frac{\partial p}{\partial y} = \frac{\partial q}{\partial x} \qquad
			\frac{\partial p}{\partial z} = \frac{\partial r}{\partial x} \qquad
			\frac{\partial q}{\partial z} = \frac{\partial r}{\partial y}
		\end{equation}
		
		\section{Green Theorem}
			\begin{equation}
				\int\limits_C P \, \mathrm{d}x + Q \, \mathrm{d}y = \iint\limits_D \left(\frac{\partial Q}{\partial x} - \frac{\partial P}{\partial y}\right) \mathrm{d}A
			\end{equation}
		
		\section{Area Calculations}
			\begin{equation}
				A = \int\limits_C x \, \mathrm{d}y = - \int\limits_C y \, \mathrm{d}x = \frac{1}{2} \int\limits_C x \, \mathrm{d}y - y \, \mathrm{d}x
			\end{equation}

		\section{Green Theorem 2}
			\begin{equation}
				\int\limits_C p \, \mathrm{d}x + q \, \mathrm{d}y = \int\limits_{C_{1}} p \, \mathrm{d}x + q \, \mathrm{d}y + \int\limits_{C_{2}} p \, \mathrm{d}x + q \, \mathrm{d}y = \iint\limits_D \left(\frac{\partial Q}{\partial x} - \frac{\partial P}{\partial y}\right) \mathrm{d}A
			\end{equation}
		
\chapter{PDF 15}\label{pdf15}
	\section{Smooth Parametrized Curve}
		\begin{equation}
			\vec{r}(u, v) = \big(x(u, v), \, y(u, v), \, z(u, v)\big), \quad (u, v) \in R
		\end{equation}
		
		\begin{equation}
			\vec{r}_u = (\frac{\partial x}{\partial u}, \, \frac{\partial y}{\partial u}, \, \frac{\partial z}{\partial u}) \qquad \vec{r}_v = (\frac{\partial x}{\partial v}, \, \frac{\partial y}{\partial v}, \, \frac{\partial z}{\partial v}) 
		\end{equation}	
		\subsection{Sample}
			Let $z = f(x, y)$, the smooth parametrized curve will be:
				\begin{equation}
					\vec{r}(u, v) = (u, v, f(u, v))
				\end{equation}
		 	Partial Derivatives will be:
			 	\begin{equation}
				 	\begin{split}
				 		\vec{r}_u 
				 		& = (
				 			\frac{\partial u}{\partial u}, \,
				 			\frac{\partial v}{\partial u}, \,
				 			\frac{\partial f(u, v)}{\partial u}
				 		) \\
				 		& = (1, 0, \frac{\partial f}{\partial u})
			              = \vec{i} + \frac{\partial f}{\partial u} \vec{k}
				 	\end{split}
			 	\end{equation}
		 	
			 	\begin{equation}
				 	\begin{split}
						\vec{r}_v 
						& = (
								\frac{\partial u}{\partial v}, \,
								\frac{\partial v}{\partial v}, \,
								\frac{\partial f(u, v)}{\partial v}
							) \\
						& = (0, 1, \frac{\partial f}{\partial v})
						  = \vec{j} + \frac{\partial f}{\partial v} \vec{k}
					\end{split}
				\end{equation}
	 	    Cross Product of the partial derivatives:
	 	    \begin{equation}
				\begin{split}
					\vec{r}_u \times \vec{r}_v 
					& = (
						- \frac{\partial f}{\partial u}, \, 
						- \frac{\partial f}{\partial v}, \, 
						1
					) \\
					& = - \frac{\partial f}{\partial u} \vec{i} + 
					    - \frac{\partial f}{\partial v} \vec{j} + 
					    \vec{k} \neq 0
				\end{split}
			\end{equation}
	\section{Area}
		\begin{equation}
			A(\mathtt{S}) = \iint\limits_R \left|\vec{r}_u \times \vec{r}_v\right| \, \mathrm{d}A
		\end{equation}
		\subsection{Theorem}
			Let $z = f(x, y)$, then:
			\begin{equation}
				A(\mathtt{S}) = 
				  \iint\limits_R 
				  \sqrt{
				  	(\frac{\partial f}{\partial x})^2 + 
				  	(\frac{\partial f}{\partial y})^2 + 
				  	1
			  	} 
				  \ \ \mathrm{d}A
			\end{equation}
\chapter{PDF 16}\label{pdf16}
\end{document}