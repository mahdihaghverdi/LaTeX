\documentclass[12pt, fleqn]{book}

\usepackage{amsmath, amssymb, amsthm, amsfonts}
\usepackage{mathtools}
\usepackage{physics}
\usepackage{hyperref}
\hypersetup{
	colorlinks=true,
	linktoc=all,
	linkcolor=blue,
}

% commands -------------------------------------------------------------------
\newcommand{\D}{\mathrm{d}}
\newcommand{\ic}{\int\limits_C}
\newcommand{\iis}{\iint\limits_S}
\newcommand{\rutrv}{\vec{r}_u \times \vec{r}_v\right}
\newcommand{\F}{\mathbf{F}}
\newcommand{\G}{\mathbf{G}}
\newcommand{\Curl}{\mathrm{curl}}
\newcommand{\Div}{\mathrm{div}}
\newcommand{\xy}{(x, y)}
\newcommand{\xyz}{(x, y, z)}
\newcommand{\uv}{(u, v)}
\newcommand{\rond}[2]{\frac{\partial #1}{\partial #2}}
% commands -------------------------------------------------------------------

\title{PDF 13 to 16 Integral Formulas}
\author{Mahdi Haghverdi}

\begin{document}
	\maketitle
	\tableofcontents
\chapter{PDF 13}\label{pdf13}
	\section{Line Integral}
	\begin{equation}
		\ic  f \, \D s = \int_{a}^{b} f\big(\vec{r}(t)\big) \, \lvert r'(t) \lvert \, \D t
	\end{equation}		

	\section{Physical Aspect}
		\begin{equation}
			m = \ic  \delta \xy \, \D s
		\end{equation}
		
		\begin{equation}
			(\bar{x}, \bar{y}) = 
			\begin{cases}
					\bar{x} = \frac{1}{m} \ic  x \, \delta\xy \, \, \D s \\
					\bar{y} = \frac{1}{m} \ic  y \, \delta\xy \, \, \D s
			\end{cases}
		\end{equation}
	
	\section{Vector Field}
		\begin{equation}
			\ic  F \cdot \D r = \int_{a}^{b} F\big(\vec{r}(t)\big) \cdot r'(t) \, \D t
		\end{equation}
		
		\begin{equation}
			\ic  F \cdot \D r = \ic  \mathtt{p} \, \D x + \mathtt{q} \, \D y + \mathtt{r} \, \D z
		\end{equation}    

\chapter{PDF 14}\label{pdf14}
    \section{Gradient Vector}
		\begin{equation}
			F = \nabla f \Rightarrow
			\begin{cases}
				p = \rond{f}{x} \\
				q = \rond{f}{y} \\
				r = \rond{f}{z} 
			\end{cases}
		\end{equation} 
		
		\begin{equation}
			\ic  F \cdot \D r = \int_{a}^{b} \nabla f \cdot \D r = f\big(\vec{r}(b)\big) - f\big(\vec{r}(a)\big)
		\end{equation}     
		
		\begin{equation}
			\rond{p}{y} = \rond{q}{x} \qquad
			\rond{p}{z} = \rond{r}{x} \qquad
			\rond{q}{z} = \rond{r}{y}
		\end{equation}
		
		\section{Green Theorem}
			\begin{equation}
				\ic  P \, \D x + Q \, \D y = \iint\limits_D \left(\rond{Q}{x} - \rond{P}{y}\right) \D A
			\end{equation}
		
		\section{Area Calculations}
			\begin{equation}
				A = \ic  x \, \D y = - \ic  y \, \D x = \frac{1}{2} \ic  x \, \D y - y \, \D x
			\end{equation}

		\section{Green Theorem 2}
			\begin{equation}
				\ic  p \, \D x + q \, \D y = \int\limits_{C_{1}} p \, \D x + q \, \D y + \int\limits_{C_{2}} p \, \D x + q \, \D y = \iint\limits_D \left(\rond{Q}{x} - \rond{P}{y}\right) \D A
			\end{equation}
		
\chapter{PDF 15}\label{pdf15}
	\section{Smooth Parametrized Curve}
		\begin{equation}
			\vec{r}\uv = \big(x\uv, \, y\uv, \, z\uv\big), \quad \uv \in R
		\end{equation}
		
		\begin{equation}
			\vec{r}_u = (\rond{x}{u}, \, \rond{y}{u}, \, \rond{z}{u}) \qquad \vec{r}_v = (\rond{x}{v}, \, \rond{y}{v}, \, \rond{z}{v}) 
		\end{equation}	
		\subsection{Sample}
			Let $z = f\xy$, the smooth parametrized curve will be:
				\begin{equation}
					\vec{r}\uv = (u, v, f\uv)
				\end{equation}
		 	Partial Derivatives will be:
			 	\begin{equation}
				 	\begin{split}
				 		\vec{r}_u 
				 		& = (
				 			\rond{u}{u}, \,
				 			\rond{v}{u}, \,
				 			\rond{f\uv}{u}
				 		) \\
				 		& = (1, 0, \rond{f}{u})
			              = \vec{i} + \rond{f}{u} \vec{k}
				 	\end{split}
			 	\end{equation}
		 	
			 	\begin{equation}
				 	\begin{split}
						\vec{r}_v 
						& = (
								\rond{u}{v}, \,
								\rond{v}{v}, \,
								\rond{f\uv}{v}
							) \\
						& = (0, 1, \rond{f}{v})
						  = \vec{j} + \rond{f}{v} \vec{k}
					\end{split}
				\end{equation}
	 	    Cross Product of the partial derivatives:
	 	    \begin{equation}
				\begin{split}
					\vec{r}_u \times \vec{r}_v 
					& = (
						\frac{- \partial f}{\partial u}, \, 
						\frac{- \partial f}{\partial v}, \, 
						1
					) \\
					& = \frac{- \partial f}{\partial u} \vec{i} +
					    (\frac{- \partial f}{\partial v} \vec{j}) + 
					    \vec{k} \neq 0
				\end{split}
			\end{equation}
	\section{Area}
		\begin{equation}
			A(\mathtt{S}) = \iint\limits_R \left|\rutrv| \, \D A
		\end{equation}
		\subsection{Theorem}
			Let $z = f\xy$, then:
			\begin{equation}
				A(\mathtt{S}) = 
				  \iint\limits_R 
				  \sqrt{
				  	(\rond{f}{x})^2 + 
				  	(\rond{f}{y})^2 + 
				  	1
			  	} 
				  \ \ \D A
			\end{equation}
	\section{Vector Integral}
		\begin{equation}
			\iis G\xyz \, \D \sigma 
			  = \iint\limits_R G\big(\vec{r}\uv\big) 
			    \left|\rutrv| 
			    \, \D A 
			    \quad \vec{r}\uv \in R
		\end{equation}
		\subsection{special cases}
			\subsubsection{1}
				Let $S: \vec{r}\uv = \big(f\uv, g\uv, h\uv\big)$, then:
			 	\begin{equation}
			 		\iis G\xyz \, \D \sigma
			 		  = \iis G\big(f\uv, g\uv, h\uv\big) 
			 		    \left|\rutrv| 
			 		    \, \D A
			 	\end{equation}
		 	\subsubsection{2}
		 		Let $z = f\xy$, then:
		 		\begin{equation}
		 			\iis G\xyz \, \D \sigma
		 			  = \iis G\big(x, y, f\xy\big)
		 			  \sqrt{
		 			  	(\rond{f}{x})^2 + 
		 			  	(\rond{f}{y})^2 + 
		 			  	1
	 			  	  } 
 			  	       \ \D x \D y
		 		\end{equation}
	 		\subsubsection{Solved Exercise (use as a template)}
	 			Let $z = \sqrt{x^2 + y^2} \ \, ; \ (0 \le z \le 1)$, then $\iis x^2 \, \D \sigma = ?$
	 			
	 			Solution:
	 			\begin{equation*}
	 				\begin{cases}
	 					x = r\cos\theta & \\
	 					y = r\sin\theta & \\
	 					z = z
	 				\end{cases}
 				    \Rightarrow
 				    \begin{cases}
	 					x = r\cos\theta & \\
						y = r\sin\theta & \\
						z= r
 				    \end{cases}
 			    	\, (0 \le r \le 1) \, ; (0 \le \theta \le 2\pi) 			    	
	 			\end{equation*}
 				\begin{equation*}
 					\Rightarrow
 					\begin{split}
 						\vec{r}\,(r, \theta) 
 						               & = (r\cos\theta, r\sin\theta, r) \\
 						\vec{r}_r      & = (\cos\theta, \sin\theta, r) \\
 						\vec{r}_\theta & = (-r\sin\theta, r\cos\theta, 0) \\
 					\end{split}
 				\end{equation*}
 				\begin{equation*}
 					\Rightarrow
 					\vec{r}_r \times \vec{r}_\theta = 
 					\begin{vmatrix}
 						i            & j             & k \\
 						\cos\theta   & \sin\theta    & r \\
 						-r\sin\theta & r\cos\theta & 0
 					\end{vmatrix} = (-r\cos\theta, -r\sin\theta, r)
 				\end{equation*}
 				\begin{equation*}
 					\Rightarrow
					\left|\vec{r}_r \times \vec{r}_\theta\right|
         			 = \sqrt{
         			 	\underbrace{
             			 	(r^2\cos^2\theta) +
            			 	(r^2\sin^2\theta)
         		 	    }_{=r^2}
         	 	        + r^2
					 }
					 = \sqrt{2r^2}
					 = r\sqrt{2} 				
 				\end{equation*}
 				\begin{equation*}
 					\begin{split}
	 					\Rightarrow \iint\limits_{[0, 1] \times [0, 2\pi]}
 						r^2\cos^2\theta \ r\sqrt{2} \ \D r \D \theta
 						 & = \sqrt{2} \int_{0}^{2\pi} \int_{0}^{1} r^3\cos^2\theta \ \D r \D \theta \\
 						 & = \frac{\sqrt{2}}{4} \int_{0}^{2\pi} \cos^2\theta \ \D \theta \\
 						 & = \frac{\sqrt{2}}{2 \times 4} \int_{0}^{2\pi} 1 + \cos2\theta \ \D \theta \\
 						 & = \frac{\sqrt{2}}{8} \left.\big(\theta + \frac{1}{2}\sin2\theta\big)\right|_0^{2\pi} \\
 						 & = \frac{2\pi \times \sqrt{2}}{8} = \frac{\pi\sqrt{2}}{4}
 					\end{split}
 				\end{equation*}
\chapter{PDF 16}\label{pdf16}
	\section{Transforms}
		\subsubsection{Del (nabla)}
			Symbol:
			\begin{equation}
				\nabla
			\end{equation}
			Definition:
			\begin{equation}
				\nabla f \coloneqq \rond{f}{x}\vec{i} + \rond{f}{y}\vec{j} + \rond{f}{z}\vec{k}
			\end{equation}
		    \begin{equation}
		    	\nabla \coloneqq \rond{}{x}\vec{i} + \rond{}{y}\vec{j} + \rond{}{z}\vec{k}
		    \end{equation}
	    \subsubsection{Divergence}
	    	Symbol:
	    	\begin{equation}
	    		\Div
	    	\end{equation}
    		Definition: \\
    		Let $\F$ be $\F\xyz = P\xyz\vec{i} + Q\xyz\vec{j} + R\xyz\vec{k}$, then:
    		\begin{equation}
    			\Div \; \F \coloneqq \rond{P}{x}\vec{i} + \rond{Q}{y}\vec{j} + \rond{R}{z}\vec{k}
    		\end{equation}
    		\begin{equation}
    			\Div \; \F \coloneqq \nabla \cdot \F
    		\end{equation}
    	\subsubsection{curl}
    		Symbol:
    		\begin{equation}
    			\Curl
    		\end{equation}
    		Definition: \\
    		Let $\F$ be $\F\xyz = P\xyz\vec{i} + Q\xyz\vec{j} + R\xyz\vec{k}$, then:
    		\begin{equation}
    			\Curl \; \F \coloneqq (\rond{R}{y} - \rond{Q}{z})\vec{i} + (\rond{P}{z} - \rond{R}{x})\vec{j} + (\rond{Q}{x} - \rond{P}{y})\vec{k} 
     		\end{equation}
     		\begin{equation}
     			\Curl \; \F \coloneqq \nabla \cdot \F = 
     			\begin{vmatrix}
     				\vec{i} & \vec{j} & \vec{k} \\
     				\rond{}{x} &  \rond{}{y} & \rond{}{z} \\
     				P & Q & R
     			\end{vmatrix}
     		\end{equation}
     	\section{Theorems}
     		\begin{enumerate}
     			\item $\Curl(\nabla f) = 0$
     			\item $\Div(\Curl \, \F) = 0$
                \item \begin{enumerate}
     					\item Let $\lambda \in \mathbb{R}$, then: $\nabla(f + \lambda g) = \nabla f + \lambda \nabla g$
		     			\item $\Curl \, (\F + \lambda \G) = \Curl \, \F + \lambda\Curl \, \G$
		     			\item $\Div \, (\F + \lambda \G) = \Div \, \F + \lambda\Div \, \G$
     			\end{enumerate}
     			\item $\Div \, (\F \times \G) = \G \cdot \Curl \, \F - \F \cdot \Curl \, \G$
     			\item $\Div \, (f \F) = f \, \Div \, \F + \F \cdot \nabla f$
     			\item $\Curl \, (f \F) = f \, \Curl \, \F + \nabla f \times \F$
     		\end{enumerate}
     	\section{Surface Integrals of Vector Fields}
     		\begin{equation}
     			\iis \F \cdot \vec{n} \ \D \sigma
     		\end{equation}
     		\begin{equation*}
     			\begin{split}
     				\iis \F \cdot \vec{n} \ \D \sigma
     				&  = \iis \F \cdot \frac{\vec{r}_u \times \vec{r}_v}{\left|\rutrv|} \ \D \sigma \\
     				& = \iint\limits_D \left[\F \cdot \frac{\vec{r}_u \times \vec{r}_v}{\left|\rutrv|}\right] \left|\rutrv| \ \D A \\
     				& = \iint\limits_D \F \cdot \left(\rutrv) \ \D A
     			\end{split}
     		\end{equation*}
     	    \begin{equation}
     	    	\Rightarrow 
     	    	\iis \F \cdot \vec{n} \ \D \sigma = \iint\limits_D \F \cdot \left(\rutrv) \ \D A
     	    \end{equation}
     		Let $z = g\xy; \, \xy \in R$ and $\F = (P, Q, R)$, then:
     		\begin{equation}
     			\iis \F \cdot \vec{n} \ \D \sigma = 
     			\iint\limits_R \left(-P \rond{g}{x} - Q\rond{g}{y} + R\right) \ \D A
     		\end{equation}
\end{document}