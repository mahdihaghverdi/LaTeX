\chapter{Books}
		\para{\textsf{You may argue me about what I want to say, however I found it true: \say{If you want to learn \textit{something}, the real source is the books about it.} This is absolutely true. No video course and no article and nothing else may contain the most valuable information about something. All the video courses and simple article ARE come from the books.}}
		
		\para{\textsf{In this chapter I may introduce many invaluable books (most technical books but it may contain other books as well.)}}
		
		\noindent\textsf{Each section of this chapter has the name of the book which is a link to the post of that book in the store: \url{Skybooks.ir}, where I buy the books from. Then there maybe some tags under the name of the books, the whole tags are categorized in three categories:}
		\sffamily
		\begin{enumerate}
			\item Reading priority
				\begin{itemize}
					\item \mustread
					\item \betteread
					\item \laterread
				\end{itemize}
			\item Buying considerations:
				\begin{enumerate}
					\item Number of pages
						\begin{itemize}
							\item \nearonethpages
							\item \nearsevenhpages
							\item \nearfivehpages
							\item \neartwohpages
						\end{itemize}
					\item Price:
						\begin{itemize}
							\item \pricenearfive{number}
							\item \pricenearfour{number}
					        \item \pricenearthree{number}
							\item \priceneartwo{number}
							\item \pricenearone{number}
						\end{itemize}
				\end{enumerate}
			\item Book read \bookread
		\end{enumerate}
		\normalfont
	    \clearpage
	    
		\section{\href{https://skybooks.ir/products/Deep-Learning-for-Coders-with-fastai-and-PyTorch}{Deep Learning for Coders With Fastai and PyTorch}}
			
			\begin{inparadesc}
				\item \mustread
				\item \nearsevenhpages
				\item \pricenearfour{297}
			\end{inparadesc}
			\vspace{3mm}
			
			\bookcover{deeplearning}
						
			\para{Deep learning is often viewed as the exclusive domain of math PhDs and big tech companies. But as this hands-on guide demonstrates, programmers comfortable with Python can achieve impressive results in deep learning with little math background, small amounts of data, and minimal code. How? With fastai, the first library to provide a consistent interface to the most frequently used deep learning applications show you how to train a model on a wide range of tasks using fastai and PyTorch.} 
			
			\para{You'll also dive progressively further into deep learning theory to gain a complete understanding of the algorithms behind the scenes. Train models in computer vision, natural language processing, tabular data, and collaborative filtering Learn the latest deep learning techniques that matter most in practice Improve accuracy, speed, and reliability by understanding how deep learning models work Discover how to turn your models into web applications Implement deep learning algorithms from scratch Consider the ethical implications of your work.}
			
			\para{\sffamily As a 622-page and comprehensive book, it needs attention and time. The book may get printed, just to be read easily.}
		
		\clearpage
		\section{\href{https://skybooks.ir/products/Fluent-Python}{Fluent Python}}
			\begin{inparadesc}
				\item \laterread
				\item \nearonethpages
				\item \pricenearfive{505}
			\end{inparadesc}
			\vspace{3mm}
			
			\bookcover{fluent}
			
			\para{Python’s simplicity lets you become productive quickly, but often this means you aren’t using everything it has to offer. With the updated edition of this hands-on guide, you’ll learn how to write effective, modern Python 3 code by leveraging its best ideas. Don’t waste time bending Python to fit patterns you learned in other languages. Discover and apply idiomatic Python 3 features beyond your past experience. Author Luciano Ramalho guides you through Python’s core language features and libraries and teaches you how to make your code shorter, faster, and more readable.} 
			
			\noindent Featuring major updates throughout the book, Fluent Python, second edition, covers:
			\begin{enumerate}
				\item Special methods: The key to the consistent behavior of Python objects
				\item Data structures: Sequences, dicts, sets, Unicode, and data classes
				\item Functions as objects: First-class functions, related design patterns, and type hints in function declarations
				\item Object-oriented idioms: Composition, inheritance, mixins, interfaces, operator overloading, static typing and protocols
				\item Control flow: Context managers, generators, coroutines, async/await, and thread/process pools
				\item Metaprogramming: Properties, attribute descriptors, class decorators, and new class metaprogramming hooks that are simpler than metaclasses
			\end{enumerate}
		
		\clearpage
		\section{\href{https://skybooks.ir/products/Build-a-Career-in-Data-Science}{Build a Career in Data Science}}
			\begin{inparadesc}
				\item \laterread
				\item \nearfivehpages
				\item \pricenearthree{194}
			\end{inparadesc}
			\vspace{3mm}
			
			\bookcover{career}
			
			\para{What are the keys to a data scientist's long-term success? Blending your technical know-how with the right "soft skills" turns out to be a central ingredient of a rewarding career.} 
			
			\para{Build a Career in Data Science is your guide to landing your first data science job and developing into a valued senior employee. By following clear and simple instructions, you'll learn to craft an amazing resumé and ace your interviews.} 
			
			\para{In this demanding, rapidly changing field, it can be challenging to keep projects on track, adapt to company needs, and manage tricky stakeholders. You'll love the insights on how to handle expectations, deal with failures, and plan your career path in the stories from seasoned data scientists included in the book.} 
			
			\noindent What's Inside:
			\begin{inparaitem}
				\item Creating a portfolio of data science projects
				\item Assessing and negotiating an offer
				\item Leaving gracefully and moving up the ladder
				\item Interviews with professional data scientistsFor readers who want to begin or advance a data science career.
			\end{inparaitem} \newline
			
			\para{Emily Robinson is a data scientist at Warby Parker.Jacqueline Nolis is a data science consultant and mentor.}
		
		\clearpage
		\section{\href{https://skybooks.ir/products/Inside-The-Python-Virtual-Machine}{Inside the Python Virtual Machine}}\label{ipvm}
			\begin{inparadesc}
				\item \betteread
				\item \neartwohpages
				\item \pricenearone{108}
			\end{inparadesc}
			\vspace{3mm}
	
			\bookcover{insidepyvenv}
						
			\para{Inside the Python Virtual Machine provides a guided tour under the covers of the Python interpreter for the curious pythonista. It attempts to show the user what happens from the moment the user executes a piece of Python code to the point when the interpreter returns the result of executing the piece of code.}
			
			\para{This book will provide the readers with an understanding of the various processes that go into compiling and executing a python program removing most of the mystery surrounding how the python interpreter executes source code.} 
			
			\para{The books starts out with a description of the compilation phase with emphasis on the less generic parts of the compilation phase. It then proceeds to discuss python objects and their implementation in CPython.  This is followed by a discussion of various objects types that are central to the interpreter such as frame objects and code objects. The process of evaluating code objects by the interpreter loop is also discussed as well as how to extend the Python programming language with your own constructs.}
			
		\clearpage
		\section{\href{https://skybooks.ir/products/CPython-Internals}{CPython Internals}}		
			\begin{inparadesc}
				\item \betteread
				\item \nearfivehpages
				\item \pricenearthree{209}
			\end{inparadesc}
			\vspace{3mm}
			
			\bookcover{internals}
			
			\para{CPython Internals: Your Guide to the Python 3 Interpreter.}
			
			\para{Are there certain parts of Python that just seem like magic? Once you see how Python works at the interpreter level, you’ll be able to optimize your applications and fully leverage the power of Python.}
			
			\para{In CPython Internals, you’ll unlock the inner workings of the Python language, learn how to compile the Python interpreter from source code, and cover what you’ll need to know to confidently start contributing to CPython yourself!}