\documentclass{article}

\usepackage{xepersian}
\settextfont{XB Yas}

\renewcommand{\arraystretch}{1.25}
\title{ترتیب برداشتن واحد}
\begin{document}
	\maketitle
	\date
	
	\begin{table}[h]
		\caption{با چُنین ترتیبی واحد‌ها را بردارید}
		\begin{center}
			\begin{tabular}{|c|c|c|c|c|}
				\hline
				ترتیب & درس & استاد & روز & کد \\
				\hline
				\hline
                \textbf{۱} 
                & معماری & بیکی & یک‌شنبه-دوشنبه & ۰۰-۰۰۰۰۰۰۰ \\				
				\hline
				\textbf{۲}
			    & آز فیزیک & لهراسبی & دوشنبه & ۰۰-۰۰۰۰۰۰۰ \\
				\hline
				\textbf{۳} 
				& نظریه زبان‌ها & کائدی & دوشنبه‌-‌چهارشنبه & ۰۰-۰۰۰۰۰۰۰ \\
				\hline
				\textbf{۴} 
				& تحلیل طراحی سیستم‌ها & شعرباف & شنبه-یک‌شنبه & ۰۰-۰۰۰۰۰۰۰ \\
				\hline
				\textbf{۵} 
				& آز منطقی & آزادی & شنبه & ۰۰-۰۰۰۰۰۰۰ \\
				\hline
				\textbf{۶} 
				& طراحی الگوریتم & کائدی & یک‌شنبه-دوشنبه & ۰۰-۰۰۰۰۰۰۰ \\
				\hline
				\textbf{۷} 
				& مدار الکتریکی & رشادی‌نژاد & شنبه-سه‌شنبه & ۰۰-۰۰۰۰۰۰۰ \\
				\hline
				\textbf{۸} 
				& آز مدار الکتریکی & خیرمند & شنبه & ۰۰-۰۰۰۰۰۰۰ \\
				\hline
			\end{tabular}
		\end{center}
	\end{table}
\end{document}