\documentclass{article}

\usepackage{multirow}
\usepackage{xepersian}
\settextfont{XB Yas}

\renewcommand{\arraystretch}{1.6}
\title{ترتیب برداشتن واحد}
\begin{document}
	\maketitle
	\date
	
	\begin{table}[h]
		\caption{با چُنین ترتیبی واحد‌ها را بردارید}
		\begin{center}
			\begin{tabular}{|c|c|c|c|c|}
				\hline
				ترتیب & درس & استاد & روز & کد \\
				\hline
				\hline
                \textbf{۱} 
                & معماری & بیکی & یک‌شنبه-دوشنبه & ‫‪۳۶۲۰۰۱۳-۰۳‬‬‬‬ \\				
				\hline
				\multirow{3}{*}{۲} &
				\multirow{3}{*}{آز فیزیک}
				& لهراسبی & دوشنبه & ‫‪۴۲۱۲۵۵۹-۱۷‬‬ \\
				\cline{3-5}
				\multirow{3}{*}{}
				& & قنبری & سه‌شنبه & ‫‪۴۲۱۲۵۵۹-۰۲‬‬ \\
				\cline{3-5}
				\multirow{3}{*}{}
				& & جلالی & دوشنبه & ‫‪۴۲۱۲۵۵۹-۰۳‬‬ \\
				\hline
				\textbf{۳} 
				& نظریه زبان‌ها & کائدی & دوشنبه‌-‌چهارشنبه & ‫‪۳۶۲۰۰۱۰-۰۱‬‬ \\
				\hline
				\textbf{۴} 
				& تحلیل طراحی سیستم‌ها & شعرباف & شنبه-یک‌شنبه & ‫‪۳۶۲۰۰۵۶-۰۱‬‬ \\
				\hline
				
				
				\multirow{3}{*}{\textbf{۵}} &
				\multirow{3}{*}{آز منطقی} &
				\multirow{3}{*}{آزادی} 
				& شنبه & ‫‪۳۶۲۰۳۱۱-۰۳‬‬ \\
				\cline{4-5}
				\multirow{3}{*}{} 
				& & & چهارشنبه & ‫‪۳۶۲۰۳۱۱-۰۱‬‬ \\
				\cline{4-5}
				\multirow{3}{*}{} 
				& & & چهارشنبه & ‫‪۳۶۲۰۳۱۱-۰۲‬‬ \\
				\hline
				
				\textbf{۶} 
				& طراحی الگوریتم & کائدی & یک‌شنبه-دوشنبه & ‫‪۳۶۲۰۰۱۵-۰۱‬‬ \\
				\hline
				\textbf{۷} 
				& آز مدار الکتریکی & خیرمند & سه‌شنبه & ‫‪‫‪۳۶۲۰۳۱۴-۰۶‬‬‬‬ \\
				\hline
				
   			\end{tabular}
		\end{center}
	\end{table}
\end{document}