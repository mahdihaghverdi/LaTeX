\documentclass{article}

\usepackage{multicol}
\usepackage{multirow}
\usepackage{rotating}
\usepackage{calc}
\usepackage[dvips=false,pdftex=false,vtex=false,b5paper]{geometry}
\geometry{
	paper=a4paper,
	inner=16mm,         % Inner margin
	outer=24mm,         % Outer margin
	bindingoffset=10mm, % Binding offset
	top=20mm,           % Top margin
	bottom=28mm,        % Bottom margin
	%showframe,          % show how the type block is set on the page
}
\renewcommand{\arraystretch}{1.9}
\usepackage{xepersian}
\settextfont{XB Yas}

\begin{document}	
	\begin{sidewaystable}[h]
		\begin{center}	
			\caption{جدول امتحانات ترم ۴}	
			\begin{tabular}{|c|c|c|c|c|c|c|c|}
				\cline{2-8}
				\multicolumn{1}{c}{} &
				\multicolumn{7}{|c|}{\textbf{هفته}} \\
				\cline{1-8}
				
				\multicolumn{1}{|c|}{\textbf{روز}} &
				\textbf{شنبه} &
				\textbf{یک‌شنبه} & 
				\textbf{دوشنبه} & 
				\textbf{سه‌شنبه} & 
				\textbf{چهارشنبه} & 
				\textbf{پنج‌شنبه} & 
				\textbf{جمعه} \\ 
				\hline \hline
				
				\multirow{7}{*}{\textbf{امتحانات}}
				& ۱۳ & ۱۴ & ۱۵ & ۱۶ & ۱۷ & ۱۸ & ۱۹ \\
				\cline{2-8}
				& & & & & & اندیشه اسلامی ۱ (۸ تا ۱۰) & آز الکتریکی (۱۹ تا ۲۰) \\
				\cline{2-8}				
				& & & & & & & دانش خانواده (۱۰ تا ۱۲) \\
				
				\cline{2-8}				
				& ۲۰ & ۲۱ & ۲۲ & ۲۳ & ۲۴ & ۲۵ & ۲۶ \\
				\cline{2-8}
				& نظریه زبان‌ها (۱۴ تا ۱۷) & & & & معماری (۱۴ تا ۱۷) & & \\
				\cline{2-8}	\cline{2-8}			
				& ۲۷ & ۲۸ & ۲۹ & ۳۰ & ۳۱ & ۰۱ & ۰۲ \\
				\cline{2-8}
				& & & تحلیل طراحی (۱۴ تا ۱۷) & & طراحی الگوریتم (۱۱ تا ۱۴) & آز منطقی (۸ تا ۱۰) & \\
				\hline
				
			\end{tabular}
		\end{center}
	\end{sidewaystable}
\end{document}